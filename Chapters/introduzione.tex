\chapter{Introduzione}
\section{Prefazione}
Questa tesi consiste nella riproduzione e nello studio di \textit{sketch-rnn} \cite{sketchrnn},
una rete neurale in grado di generare disegni di semplici oggetti, composti da 
sequenze di tratti, appresi da un dataset di disegni creati da esseri umani. Il 
dataset è costantemente ampliato tramite \textit{Quick Draw!} \cite{quickdraw}, 
un gioco online in cui agli utenti viene chiesto di disegnare alcuni oggetti entro 
20 secondi, che al momento della stesura di questa tesi costituisce la più 
vasta collezione di disegni al mondo. In questo lavoro viene proposta 
un'implementazione in \textit{Keras} \cite{keras}, un framework che a sua 
volta poggia su \textit{tensorflow} \cite{tensorflow}, che è la libreria 
utilizzata per il lavoro originale.
\section{Stato dell'arte}
\section{Lavori correlati}
\section{obbiettivi del progetto}