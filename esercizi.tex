\chapter{esercizi}

\begin{enumerate}
\medskip
\item 
Scrivere le possibili evoluzioni del programma 
\begin{verbatim}
                       		co X: = X+2 // X: = X+1 oc
\end{verbatim}
assumendo che ciascun assegnamento � realizzato da tre azioni atomiche che
caricano X in un registro (\verb"Load R X"), incrementano il valore del registro (\verb"Add R v") e memorizzano il valore del registro((\verb"Store R X"). Per ciascuna delle esecuzioni risultanti dall�interleaving delle azioni atomiche descrivere il contenuto dopo ogni passo della locazione condivisa X e dei registri privati,  $R_1$ del processo che esegue il primo assegnamento ed  $R_2$  per il processo che esegue il secondo assegnamento. Se assuma che il valore iniziale di X sia 50.
\medskip
\item
Si definisca il problema della \emph{barrier synchronization} e si descrivano per sommi capi i differenti approcci alla sua soluzione. Se ne fornisca quindi una soluzione dettagliata utilizzando i semafori.
\medskip
\item
Considerare n api ed un orso che possono avere accesso ad una tazza di miele
inizialmente  vuota e con una capacit� di k porzioni. L'orso dorme finch� la
tazza � piena di k-porzioni,  quindi mangia tutto il miele e si rimette a
dormire. Le api riforniscono in continuazione la  tazza con una porzione di
miele finch� non si riempie; l'ape che aggiunge la k-esima porzione sveglia
l'orso. Fornire una soluzione al problema modellando orso ed api come  processi
e utilizzando un monitor per gestire le loro operazioni sulla tazza. Prevedere che le api possano eseguire l'operazione \emph{produce-honey} anche concorrentemente.
 \medskip
 \item  Descrivere le primitive di scambio messaggi send e receive sia sincrone che asincrone ed implementare
 \vspace{-0.5cm}
 \begin{verbatim}
       - synch_send(v:int)
       - send(v:int)
       - receive(x:int)
\end{verbatim}
 \vspace{-0.3cm}
utilizzando le primitive di LINDA.

\end{enumerate}
